\documentclass[12pt,a4paper]{article}
\usepackage[utf8]{inputenc}
\usepackage[T1]{fontenc}
\usepackage{amsmath}
\usepackage{amsfonts}
\usepackage{amssymb}
\usepackage{graphicx}
\usepackage{tikz}
\usepackage{titlecaps}
\usepackage{float}
\usepackage{setspace}
\usepackage{wrapfig}

\usepackage[french]{nomencl}
\usepackage{french}
\usepackage{listings}
\usepackage{verbatim}
\usepackage{hyperref} 
\usepackage{pdfpages}
\usepackage{fancyhdr}

\usepackage{color}
\definecolor{lightgray}{rgb}{.9,.9,.9}
\definecolor{darkgray}{rgb}{.4,.4,.4}
\definecolor{purple}{rgb}{0.65, 0.12, 0.82}

\lstset{
	language=C,
	backgroundcolor=\color{lightgray},
	extendedchars=true,
	basicstyle=\footnotesize\ttfamily,
	showstringspaces=false,
	showspaces=false,
	tabsize=2,
	breaklines=true,
	showtabs=false,
	captionpos=b
}


\author{
	Boubekri, Abdelmalek \     \texttt{abdelmalek.boubekri@etu.univ-rouen.fr}
	\and
	Mezheri, Bilal\   		   \texttt{bilal.mezheri@etu.univ-rouen.fr}
}


\title{	Université de Rouen, UFR Sciences et Techniques\\
		Master 1 SSI – Compilation\\
		M2L, traducteur \LaTeX en MathML\\ }


\begin{document}
	
	\maketitle
	\newpage

\section{Introduction}
Le projet consiste à réaliser un traducteur \LaTeX en MathML, le programme devra prendre les formules entourées du symbole \$ et produire le code MathML correspondant, et doit prendre en compte les opérateurs, identifiants, les constantes numériques, 

\section{Grammaire}
\begin{tabbing}
FORMULE 	$\rightarrow$\= \textbackslash sqrt RACINE FORMULE\\
	\>|\textbackslash frac FRACTION FORMULE\\
	\>| UNDEROVER FORMULE\\
	\>| CARACTERE FORMULE\\
	\>| OPERATION FORMULE\\
	\>| $\epsilon$ \\

RACINE	~~~~$\rightarrow$ [FORMULE]\{FORMULE\} \\
		\>| \{FORMULE\}\\

\linebreak

FRACTION	$\rightarrow$ \{FORMULE\}\{FORMULE\}\\

\linebreak

OPERATION	$\rightarrow$ $\wedge$ FORMULE\\
		\>| \_ FORMULE\\

\linebreak

UNDEROVER	$\rightarrow$ \textbackslash sum\_\{FORMULE\} $\wedge$ \{FORMULE\} \\
		\>| \textbackslash prod\_\{FORMULE\} $\wedge$ \{FORMULE\} \\

\linebreak

CARACTERE	$\rightarrow$ IDENTIFIER | OPERATOR | NUMBER\\
 \>| \textbackslash pm | \textbackslash alpha | \textbackslash beta | \textbackslash infty |  \textbackslash mp \\
 \>| \textbackslash cap | \textbackslash cup  | \textbackslash subset  | \textbackslash supset  | \textbackslash subseteq | \textbackslash supseteq

\end{tabbing}



\section{Utilisation}

Récupérez l'archive depuis Github: https://github.com/kabyliano/l2m.git
Afin de compiler le programme il suffit de lancer le \textbf{Makefile}

\begin{lstlisting}[language=Bash]
Make
\end{lstlisting}
Puis vous pourrez lancer le programme \textbf{l2m} avec comme argument le flux \LaTeX, exemple:

\begin{lstlisting}[language=Bash]
./l2m "\$\frac{5}{9}$"
\end{lstlisting}

\section{Problèmes non résolus}
\begin{itemize}
	\item Ereur de segmentation (core dumped) dû à la mauvaise gestion de la mémoire.
	\item Récursivité mal implémentée.
	\item Conflit Décalage/Réduction (\textit{Shift/reduce}) dû aux conflits entre \_ et $\wedge$ (POW / UND) et SUM / PROD.
\end{itemize}

	
\end{document}
